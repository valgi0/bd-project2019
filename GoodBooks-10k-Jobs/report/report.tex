%%
%% Author: lorenzo and Riccardo
%% 7/19/19
%%

% Preamble
\documentclass[11pt]{article}

% Packages
\usepackage{amsmath}
\usepackage{hyperref}

\title{\textbf{Report on Big Data project: GoodBooks-10K-Jobs}}

\author{
Lorenzo Valgimigli - Mat.\\
Riccardo Soro - Mar. }

\date{\today}
% Document
\begin{document}

\maketitle
\newpage

\tableofcontents

\newpage

\section{Introduzione}
\subsection{Descrizione Dataset}

Il dataset scelto viene fornito dal negozio online di libri \href{https://www.goodreads.com/}{Good Reads} ed è
reperibile sul repository github \href{https://github.com/zygmuntz/goodbooks-10k}{GoodBooks-10k}.
All'interno possiamo trovare una serie di file \texttt{.csv} che contengono 10000 libri con vari attributi,
oltre 6 milioni di ratings e qualche centinaia di bookmarks.
In totale il data set raggiunge circa 95 MB. Colleziona 10000 libri, 53424 utenti e
oltre 6 milioni di rating.

\subsubsection{Descrizione File}
Il dataset è composto dai seguenti file:
\begin{itemize}
    \item \texttt{books.csv}: [lvalgimigli@isi-vclust9.csr.unibo.it:./exam/dataset/books.csv] Contiene metadati dei vari libri come:
    \begin{itemize}
        \item \texttt{book\_id}: ID del libro nel dataset
        \item \texttt{goodreads\_book\_id}: ID del libro nel Data Store del negozio
        \item \texttt{best\_book\_id}: ID della versione del libro che il negozio ha in vendita
        \item \texttt{work\_id}: ID per gastire dove il libro si trova nel magazzino
        \item \texttt{books\_count}: Numero di libri nel magazzino
        \item \texttt{isbn}
        \item \texttt{isbn13}
        \item \texttt{authors}: Lista di autori separati da una virgola
    \end{itemize}
    \item \texttt{ratings.csv}: [lvalgimigli@isi-vclust9.csr.unibo.it:./exam/dataset/ratings.csv] Contiene i rating fatti da tutti gli utenti. Le colonne di questa tabella sono:
    \begin{itemize}
        \item \texttt{user\_id}: ID dell'user che ha fatto la votazione
        \item \texttt{book\_id}: ID del libro nel dataset
        \item \texttt{rating}: Valore da 1 a 5 che rappresenta il rating dell'utente per quel libro
    \end{itemize}
    \item \texttt{to\_read.csv}: [lvalgimigli@isi-vclust9.csr.unibo.it:./exam/dataset/to\_read.csv] contiene i desideri degli utenti di leggere un libro. Questo desiderio è espresso con un \textit{bookmark}
            la tabella è così composta:
    \begin{itemize}
        \item \texttt{user\_id}
        \item \texttt{book\_id}
    \end{itemize}
    \item \texttt{book\_tags.csv}: [lvalgimigli@isi-vclust9.csr.unibo.it:./exam/dataset/book\_tags.csv] Questa tabella contiene per ogni libro i tags assegnatoli dagli utenti
    \begin{itemize}
        \item \texttt{goodreads\_book\_id}
        \item \texttt{tag\_id}: ID del tag
        \item \texttt{count}: Numero di utenti che hanno assegnato quel tag a quel libro
    \end{itemize}
    \item \texttt{tags.csv}: [lvalgimigli@isi-vclust9.csr.unibo.it:./exam/dataset/tags.csv] Tabella che salva l'associazione tra tag ID e il nome del tag
    \begin{itemize}
        \item \texttt{tag\_id}
        \item \texttt{tag\_name}
    \end{itemize}
\end{itemize}



\section{Preparazione dei dati}

Il referente del gruppo è lo studente Lorenzo Valgimigli. Tutti i file saranno presenti nella sua home.

Il dataset per l'esame si trova nel nodo \texttt{vclust9} nella home di \texttt{lvalgimigli} %%todo add soro
all'indirizzo \texttt{isi-vclust9.csr.unibo.it//home/lvalgimigli/exam/dataset/}
I dati venivano indicati dal sito come completi e puliti per tanto non è stata necessaria una ulteriore fase di
pulizia dati. E' stato solo sufficente dedicare del tempo a capire il dominio e il significato degli attributi
delle varie tabelle.


\section{Jobs}

L'obbiettivo del progetto è quello di realizzare 2 jobs utilizzando sia \texttt{Hadoop MapReduce} che
\texttt{Spark}. In quest'ultimo caso è stato scelto di utilizzare \texttt{SparkSql}. Di seguito
vengono esposti i Job:
    \begin{enumerate}
        \item \textbf{Job1}: Trovare i 500 migliori autori nel dataset. Ovvero, trovare i 500 autori la cui media di punteggio
            ottenuto nei loro libri è più alto. Per tanto sarà necessario calcolare per ogni libro il rating medio.
            Poi sarà necessario associare i libri agli autori e calcolare la media del rating dei vari libri.
        \item \textbf{Job2}:Trovare una correlazione tra numero di \textit{bookmarks} e \textit{rating medio} dei vari libri.
        Sarà quindi necessario calcolare per ogni libro i bookmarks e il rating medio.
    \end{enumerate}

E' bene sottolineare che già dalla fase di analisi dei requisiti si evince un problema che dovrà essere gestito: il campo
\texttt{authors} in \texttt{books.csv} è composto da una lista di autori, per tanto sarà necessario dividerli.
Si è scelto di creare un progetto \texttt{gradle} usando \texttt{Java} per la parte di
\texttt{Hadoop MapReduce} mentre si è scelto di utilizzare \texttt{Scala} per la parte di \texttt{SparkSQL}

\newpage
\section{Job1: Best 500 Authors}

In questo primo esercizio lo scopo consiste nel trovare i 500 autori che hanno le medie delle valutzioni più alte.
I punteggi vanno da 1 a 5 e sono assegnati dagli utenti ai libri, mentre per ogni libro è presente la lista di autori che
partecipato alla sua scrittura.

\subsection{Pianificazione}
Le tabelle che entrano in gioco per questo job sono le seguenti:
\begin{itemize}
    \item \texttt{ratings.csv}
    \item \texttt{books.csv}
\end{itemize}

%% todo cosa servono le tabelle
La tabella ratings è composta dall'id dell'utente, l'id del libro e dalla valutazione, quest'ultima varia da 1 a 5.
La chiave primaria in questa tabella è la combinazione id_libro e id_utente e ci serve per calcolare la media delle
valutazioni per ogni libro e, successivamente, calcolare la media del punteggio medio di ogni libro scritto da ogni attore.
Per fare questa operazione si è resa necessaria la tabella books.csv, nella quale sono presenti gli autori che hanno
partecipato alla scrittura dei libri.
In questa seconda tabella sono presenti moltissimi attributi, ma gli unici che servono sono la coppia id_book
(chiave primaria della tabella) e la lista di autori (separati da ;).

%% todo: veloce ragionamento risolutivo del problema
Per risolvere il problema occorre come prima cosa effettuare un join tra le due tabelle sull'attributo id_book, comune ad entrambe.
Occorre inoltre calcolare la media delle recenzioni di ogni libro, per poi dare un punteggio ad ogni autore pari alla media delle
valutazioni di tutti i libri sui quali ha lavorato.
Una volta ottenuta una tabella con attributi autore - rating_medio manca solo la fase di sorting per ordinare i risultati.

\subsection{Hadoop MapReduce}
Per svolgere l'esercizio è stato necessario implementare più job, è stato possibile ridurne il numero fino a tre job distinti tra loro.
Nello specifico il primo di questi si occupa di fare il join tra le due tabelle, il secondo svolge un operazione di raggruppamento e l'ultimo
ordina correttamento e globalmente gli elementi.

\begin{itemize}
    \item \textbf{}: Fase di Join : Per questa fase è stato definito un job che esegua il join delle due tabelli sulla chiave
    id_book. Questo job è composto da due Mapper e un Reducer, i mapper (uno a tabella) si occupa di leggere la tabella, escludere
    gli attributi senza utilità e di creare come output una lista di coppie id_book - attributo.
    Dalla tabella books.csv viene estrapolato l'attributo authors e dalla seconda il rating; il reducer si occupa invece
    di raccogliere questi output, calcolare la media delle votazioni di ogni libro per poi produrre un output composto dalla
    lista di coppie contenenti singolo_autore - rating_medio_per_singolo_libro. L'output perciò potrà contenere più volte lo stesso
    autore, nello specifico ogni autore sarà presente tante volte quante sono le volte in cui appare come collaboratore nei libri.

    \item \textbf{}: Fase di Filtering : Nella fase precedente si sono ottenuti in output risultati incompleti, con autori che si
    ripetono e con valori di valutazione diversi. La fase di Filtering ha come obiettivo quello di ottenere un output
    completo di tutte le informazioni richieste dall'esercizio senza avere però informazioni ridondanti o non necessarie.
    L'implementazione di questa fase è composta da da un mapper e un reducer, il primo elabora l'output della fase precedente creando
    in uscita una lista contenente autore - rating_medio_per_singolo_libro, la quale sarà ulteriormente elaborata dal reducer al fine di ottenere
    per ogni autore la media delle medie dei libri da lui scritti.
    Grazie a questa fase si ottiene in output tutti i valori necessari alla risoluzione dell'esercizio, ma essendo l'output suddiviso in più file
    (uno a reducer) non possiamo in questa fase ottenere un output ordinato globalmente.

    \item \textbf{}: Fase di Sorting : Nella fase precedente si sono ottenuti come output risultati disordinati globalmente.
    Per risolvere un ploblema di sorting globale è stato utilizzato inizialmente un partitioner personalizzato, basato su soglie del rating_medio
    che garantissero una distribuzione omogenea dei dati sul reducer garantendo l'ordinamento grazie appunto alle soglie prestabilite.
    Questa implementazione però non poteva funzionare con un numero variabile di reducer, il numero di possibili soglie utilizzabili erano limitate;
    è stato perciò deciso di utilizzare il TotaOrderPartitioner.
    Questo metodo calcola in automatico le soglie partendo da un numero specificabile di recond, permettendo al job di poter essere eseguito su un numero qualsiasi
    di reducer per non limitare la scalabilità della soluzione.
    In uscita da questo job si ottiene quindi la lista degli autori e del loro rating_medio ordinati dal rating più alto a quello più basso.
    



    \end{itemize}

\subsection{SparkSQL}

\subsection{Conclusioni}


\newpage
\section{Job2: Correlazione tra rating e bookmarks}

L'obbiettivo di questo Job è quello di vedere se esiste una correlazione tra rating medio e bookmarks
Quindi il cuore del Job è il calcolo del rating e dei bookmarks e fare un confronto incrociato.

\subsection{Pianificazione}
Le tabelle che entrano in gioco per questo job sono le seguenti:
\begin{itemize}
    \item \texttt{ratings.csv}
    \item \texttt{to\_read.csv}
\end{itemize}

La prima verrà utilizzata per calcolare il rating medio mentre la seconda verrà utilizzata per calcolare i Bookmarks.
Dopo di che i risultati ottenuti saranno uniti in una unica tabella e da li verranno calcolati i risultati.

Ogni libro può quindi appartenere a uno dei 4 gruppi:
\begin{itemize}
    \item \textit{Molti Bookmarks e alto rating}
    \item \textit{Molti Bookmarks e basso rating}
    \item \textit{Pochi Bookmarks e basso rating}
    \item \textit{Pochi Bookmarks e alto rating}
\end{itemize}

\subsection{Hadoop MapReduce}

Per risolvere l'esercizio sono state calcolate empiricamente le due soglie da utilizzare per classificare i libri.
Una volta ottenute queste soglie è stato possibile pianificare il numero di job necessari e il loro funzionamento.
Il problema si può dividere in quattro fasi:

\begin{itemize}
    \item \textbf{Calcolare il rating} In questa fase bisogna leggere la tabella rating, eliminare gli attributi non necessari
    e calcolare la media delle valutazioni per ogni libro.
    \item \textbf{Calcolare i bookmarks} In questa fase bisogna leggere la tabella to_read, eliminare gli attributi non necessari
    e calcolare la somma del numero degli utenti che desiderano leggere il libro,tutto questo per ogni libro.
    \item \textbf{Classificare per rating} In questa fase si classificano i libri comparando il rating calcolato
    precedentemente con la soglia ricavata empiricamente.
    \item \textbf{Classificare per bookmarks}In questa fase si classificano i libri comparando il numero degli utenti che li desiderano leggere
    con la soglia ricavata empiricamente.
     \item \textbf{Conteggio delle classi}In questa fase si conta il numero di record delle varie classi al fine di determinarne le più
     frequenti.
\end{itemize}

In fase di implementazione è stato possibile raggruppare questi step al fine di ottenere un numero minore di job, di conseguenza raggiungere
livelli di performance migliori.
L'esercizio è stato svolto aggregando le prime quattro fasi nel job computeAVGRatingJob,
mentre l'ultima è stata implmentata in counterJob.

\begin{itemize}
    \item \textbf{computeAVGRatingJob} Questo job legge entrambe le tabelle e realizza le fasi uno e due, per poi fare i calcoli delle fasi tre e quattro.
    Il jo si conclude con la creazione dell'output, composta dalla semplice classe di appartenenza di ogni libro.
    \item \textbf{counterJob} Questo job si occupa della lettura del file generato nel job precedente e di contare i valori in esso contenuto al
    fine di creare come file di output un file contenente il totale di libri per ogni classe.
\end{itemize}

\subsubsection{Implementazione}
Per implementare il primo job si è reso necessario lo sviluppo di due mapper, uno per ogni tabella.
Entrambi i mapper leggono dal file gli attributi rilevanti escludendo quelli non necessari al completamento dell'esercizio.
Nello specifico il mapper relativo alla tabella rating.csv crea come output una lista di book_id - rating per ogni libro, mentre l'altro
produce una lista di book_id per ogni volta che quel libro è presente nella lista dei libri da leggere di un utente.
Il reducer che è incaricato della gestione dell'output di questi due mapper è CounterRatingReducer, il quale si occupa anche della classificazione dei
libri creando un output contenente la classe di appartenenza di ogni libro.

Il secondo job si occupa semplicemente di contare il numero di libri che vengono classificati nelle quattro possibili classi.
Il mapper utilizzato da questo job è CountMapper e come chiave dell'output utilizza la classe di appartenenza del libro.
Il reducer utilizzato è CountReducer e scrive sul file di output il numero di volte che un libro è presente in una classe.
Per questo secondo job è possibile sfruttare al massimo quattro reducer, la causa è il numero di classi che possono avere i libri.
Il mapper infatti può produrre al massimo quattro tipi diversi di chiave, di conseguenza qualsiasi funzione si utilizza per il partizionamento
creerà al massimo quattro risultati diversi, limitando di fatto il numero di reducer che si possono sfruttare.


\subsection{SparkSQL}
Per questo problema si \è utilizzato a pieno SparkSQL affidandoci alle ottimizzazioni interne del framework. Il problema
\è stato affrontato tramite i seguenti step:
\begin{itemize}
    \item \texttt{Creazione DataFrame}. Come primo passo abbiamo sfruttato una feature di \texttt{spark2} per leggere
    i dati dal file .csv inferendo direttamente lo schema della tabella. Una volta letto il file viene automaticamente creato
    un oggetto DataFrame pronto all'uso.
    \item \texttt{Rating medio}. Per calcolare il rating medio non abbiamo bisogno di fare Join con altre tabelle ma ci basta utilizzare
    ratings. E' stata quindi utilizzata la seguente query per calcolare il rating:
    \begin{verbatim}
    SELECT book_id, AVG(rating) as avgRating FROM ratings
    GROUP BY book_id
    \end{verbatim}
    \item \texttt{Bookmarks}. Anche per calcolare i bookmarks bancia lanciare una query su una sola tabella:
    \begin{verbatim}
        SELECT book_id, count(user_id) as marks FROM bookmarks
        GROUP BY book_id
    \end{verbatim}
    \item \texttt{Join tra i risultati}. Ora bisogna unire i due risultati in una unica tabella. Per il join essendo le
    tabelle abbastanza piccole è stato usato un \texttt{broadcast join} sulla tabella con i rating.
    \item \texttt{Risultati}. Ora abbiamo una tabella con tutte le informazioni che ci servono.
    Ci basterà lanciare qualche query che filtra i risultati per rating e bookmarks e vedere quanti elementi sono in
    ogni gruppo.
\end{itemize}
\subsection{Conclusioni}

Realizzare questo task in SparkSQL è stato estremamente semplice sià per la semplicità in se del task sia per l'interfacca
intuitiva di SparkSQL. Il risultato finale viene calcolato in tempo abbastanza buoni grazie alle ottimizzazioni interne del framework e
al setup generale. \href{ } %%todo set up spark sql




\newpage
    
\section{Miscellaneous}

\subsection{Problematiche ottimizzazione}

Il problema che abbiamo riscontrato per quanto concerne l'ottimizzazione è la dimensione del DataSet.
Il dataset è di dimensioni relativamente ridotte: sotto i 100 MB. Per tanto l'ottimizzazione fatta dal
framework SparkSQL di default è già molto buona. Sono stati provati vari settaggi per provare ad apportare delle migliorie
ma la piccola dimensione non ha permesso l'apprezzamento di sostanziali differenze.

\subsection{Problematica Job1}

Per lo sviluppo del Job1 utilizzando il modello MapReduce non si è considerato opportuno allocare un intero Job per
selezionare i primi 500 autori, infatti sarebbe stata un'operazione pesante per il modello MapReduce mentre un semplice script
la potrebbe farlo con un costo computazionale decisamente più piccolo. Ordinarli secondo il rating medio è stato considerato sufficente.




\end{document}