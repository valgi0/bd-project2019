%%
%% Author: lorenzo and Riccardo
%% 7/19/19
%%

% Preamble
\documentclass[11pt]{article}

% Packages
\usepackage{amsmath}
\usepackage{hyperref}

\title{\textbf{Report on Big Data project: GoodBooks-10K-Jobs}}

\author{
Lorenzo Valgimigli - Mat.
Riccardo Soro - Mar. }

\date{\today}
% Document
\begin{document}

\maketitle

\newpage

\tableofcontents

\newpage

\section{DataSet Overview}

Il dataset scelto viene fornito dal negozio online di libri \href{https://www.goodreads.com/}{Good Reads} ed è
reperibile sul repository github \href{https://github.com/zygmuntz/goodbooks-10k}{GoodBooks-10k}.

All'interno possiamo trovare una serie di file \texttt{.csv} che contengono 10000 libri con vari attributi,
oltre 6 milioni di ratings e qualche centinaia di bookmarks. Ma vediamo nel dettaglio:
\begin{itemize}
    \item \texttt{books.csv}: Contiene metadati dei vari libri come:
    \begin{itemize}
        \item \texttt{book\_id}: ID del libro nel dataset
        \item \texttt{goodreads\_book\_id}: ID del libro nel Data Store del negozio
        \item \texttt{best\_book\_id}: ID della versione del libro che il negozio ha in vendita
        \item \texttt{work\_id}: ID per gastire dove il libro si trova nel magazzino
        \item \texttt{books\_count}: Numero di libri nel magazzino
        \item \texttt{isbn}
        \item \texttt{isbn13}
        \item \texttt{authors}: Lista di autori separati da una virgola
    \end{itemize}
    \item \texttt{ratings.csv}: Contiene i rating fatti da tutti gli utenti. Le colonne di questa tabella sono:
    \begin{itemize}
        \item \texttt{user_id}: ID dell'user che ha fatto la votazione
        \item \texttt{book_id}: ID del libro nel dataset
        \item \texttt{rating}: Valore da 1 a 5 che rappresenta il rating dell'utente per quel libro
    \end{itemize}
    \item \texttt{to_read.csv}: contiene i desideri degli utenti di leggere un libro. Questo desiderio è espresso con un \textit{bookmark}
            la tabella è così composta:
    \begin{itemize}
        \item \texttt{user_id}
        \item \texttt{book_id}
    \end{itemize}
    \item \texttt{book_tags.csv}: Questa tabella contiene per ogni libro i tags assegnatoli dagli utenti
    \

\end{itemize}
\end{itemize}






\end{document}