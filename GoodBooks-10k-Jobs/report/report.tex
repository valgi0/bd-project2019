%%
%% Author: lorenzo and Riccardo
%% 7/19/19
%%

% Preamble
\documentclass[11pt]{article}

% Packages
\usepackage{amsmath}
\usepackage{hyperref}

\title{\textbf{Report on Big Data project: GoodBooks-10K-Jobs}}

\author{
Lorenzo Valgimigli - Mat.
Riccardo Soro - Mar. }

\date{\today}
% Document
\begin{document}

\maketitle

\newpage

\tableofcontents

\newpage

\section{DataSet Overview}

Il dataset scelto viene fornito dal negozio online di libri \href{https://www.goodreads.com/}{Good Reads} ed è
reperibile sul repository github \href{https://github.com/zygmuntz/goodbooks-10k}{GoodBooks-10k}.
All'interno possiamo trovare una serie di file \texttt{.csv} che contengono 10000 libri con vari attributi,
oltre 6 milioni di ratings e qualche centinaia di bookmarks. Ma vediamo nel dettaglio:
\begin{itemize}
    \item \texttt{books.csv}: Contiene metadati dei vari libri come:
    \begin{itemize}
        \item \texttt{book\_id}: ID del libro nel dataset
        \item \texttt{goodreads\_book\_id}: ID del libro nel Data Store del negozio
        \item \texttt{best\_book\_id}: ID della versione del libro che il negozio ha in vendita
        \item \texttt{work\_id}: ID per gastire dove il libro si trova nel magazzino
        \item \texttt{books\_count}: Numero di libri nel magazzino
        \item \texttt{isbn}
        \item \texttt{isbn13}
        \item \texttt{authors}: Lista di autori separati da una virgola
    \end{itemize}
    \item \texttt{ratings.csv}: Contiene i rating fatti da tutti gli utenti. Le colonne di questa tabella sono:
    \begin{itemize}
        \item \texttt{user\_id}: ID dell'user che ha fatto la votazione
        \item \texttt{book\_id}: ID del libro nel dataset
        \item \texttt{rating}: Valore da 1 a 5 che rappresenta il rating dell'utente per quel libro
    \end{itemize}
    \item \texttt{to\_read.csv}: contiene i desideri degli utenti di leggere un libro. Questo desiderio è espresso con un \textit{bookmark}
            la tabella è così composta:
    \begin{itemize}
        \item \texttt{user\_id}
        \item \texttt{book\_id}
    \end{itemize}
    \item \texttt{book\_tags.csv}: Questa tabella contiene per ogni libro i tags assegnatoli dagli utenti
    \begin{itemize}
        \item \texttt{goodreads\_book\_id}
        \item \texttt{tag\_id}: ID del tag
        \item \texttt{count}: Numero di utenti che hanno assegnato quel tag a quel libro
    \end{itemize}
    \item \texttt{tags.csv}: Tabella che salva l'associazione tra tag ID e il nome del tag
    \begin{itemize}
        \item \texttt{tag\_id}
        \item \texttt{tag\_name}
    \end{itemize}
\end{itemize}

In totale il data set raggiunge circa 95 MB. Colleziona 10000 libri, 53424 utenti e
oltre 6 milioni di rating.

\section{Target}

L'obbiettivo del progetto è quello di realizzare 2 jobs utilizzando sia \texttt{Hadoop MapReduce} che
\texttt{Spark}. In quest'ultimo caso è stato scelto di utilizzare \texttt{Spark-Sql}. Di seguito
vengono esposti i Jobs:
    \begin{enumerate}
        \item \textbf{Job1}: Trovare i 500 migliori autori nel dataset. Ovvero, trovare i 500 autori la cui media di punteggio
            ottenuto nei loro libri è più alto. Per tanto sarà necessario calcolare per ogni libro il rating medio.
            Poi sarà necessario associare i libri agli autori e calcolare la media del rating dei vari libri.
        \item \textbf{Job2}:Trovare una correlazione tra numero di \textit{bookmarks} e \textit{rating medio} dei vari libri.
        Sarà quindi necessario calcolare per ogni libro i bookmarks e il rating medio.
    \end{enumerate}

E' bene sottolineare che già dalla fase di analisi dei requisiti si evince un problema che dovrà essere gestito: il campo
\texttt{author} in \texttt{books.csv} è composto da una lista di autori per tanto sarà necessario dividerli.
Per sviluppare il progetto si è scelto di creare un progetto \texttt{gradle} usando \texttt{Java} per la parte di
\texttt{Hadoop MapReduce} mentre si è scelto di utilizzare \texttt{Scala} per la parte di \texttt{SparkSQL}

\section{Job2: Correlazione tra rating e bookmarks}

L'obbiettivo di questo Job è quello di vedere se esiste una correlazione tra rating medio e bookmarks
Quindi il cuore del Job è il calcolo del rating e dei bookmarks e fare un confronto incrociato.

\subsection{Pianificazione}
Le tabelle che entrano in gioco per questo job sono le seguenti:
\begin{itemize}
    \item \texttt{ratings.csv}
    \item \texttt{to\_read.csv}
\end{itemize}

La prima verrà utilizzata per calcolare il rating medio mentre la seconda verrà utilizzata per calcolare i Bookmarks.
Dopo di che i risultati ottenuti saranno uniti in una unica tabella e da li verranno calcolati i risultati.

Ogni libro può quindi appartenere a uno dei 4 gruppi:
\begin{itemize}
    \item \textit{Molti Bookmarks e alto rating}
    \item \textit{Molti Bookmarks e basso rating}
    \item \textit{Pochi Bookmarks e basso rating}
    \item \textit{Pochi Bookmarks e alto rating}
\end{itemize}

\subsection{Hadoop MapReduce}

Per prima cosa bisogna definire i requisiti in un modello MapReduce.
\begin{itemize}
    \item \textbf{Rating medio}: %%todo breve descrizione di come implementare su mapreduce
    \item \textbf{Conteggio BookMarks}: %%todo
\end{itemize}

\subsubsection{Implementazione}
%% todo come è stato implementato quindi qunati mapper, quanti reducer, ziffatelle per farlo più veloce
%% todo considerazioni sulle performance, quali sono i punti più costosi e come sono stati ottimizzati
\subsection{SparkSQL}
Per questo problema si \è utilizzato a pieno SparkSQL affidandoci alle ottimizzazioni interne del framework. Il problema
\è stato affrontato tramite i seguenti step:
\begin{itemize}
    \item \texttt{Creazione DataFrame}. Come primo passo abbiamo sfruttato una feature di \texttt{spark2} per leggere
    i dati dal file .csv inferendo direttamente lo schema della tabella. Una volta letto il file viene automaticamente creato
    un oggetto DataFrame pronto all'uso.
    \item \texttt{Rating medio}. Per calcolare il rating medio non abbiamo bisogno di fare Join con altre tabelle ma ci basta utilizzare
    ratings. E' stata quindi utilizzata la seguente query per calcolare il rating:
    \begin{verbatim}
    SELECT book_id, AVG(rating) as avgRating FROM ratings
    GROUP BY book_id
    \end{verbatim}
    \item \texttt{Bookmarks}. Anche per calcolare i bookmarks bancia lanciare una query su una sola tabella:
    \begin{verbatim}
        SELECT book_id, count(user_id) as marks FROM bookmarks
        GROUP BY book_id
    \end{verbatim}
    \item \texttt{Join tra i risultati}. Ora bisogna unire i due risultati in una unica tabella. Per il join essendo le
    tabelle abbastanza piccole è stato usato un \texttt{broadcast join} sulla tabella con i rating.
    \item \texttt{Risultati}. Ora abbiamo una tabella con tutte le informazioni che ci servono.
    Ci basterà lanciare qualche query che filtra i risultati per rating e bookmarks e vedere quanti elementi sono in
    ogni gruppo.
\end{itemize}
\subsection{Conclusioni}

Realizzare questo task in SparkSQL è stato estremamente semplice sià per la semplicità in se del task sia per l'interfacca
intuitiva di SparkSQL. Il risultato finale viene calcolato in tempo abbastanza buoni grazie alle ottimizzazioni interne del framework e
al setup generale. \href{ } %%todo set up spark sql







\end{document}