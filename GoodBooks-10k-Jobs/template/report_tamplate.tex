\documentclass[10pt]{article}
\usepackage[utf8]{inputenc}
\usepackage{hyperref}
\usepackage[pdftex]{graphicx}


\title{\textbf{Report on Big Data project: \\(short subtitle)}}

\author{
Name Surname - Mat. 004815\\
Name Surname - Mat. 162342}

\date{\today}

\begin{document}
    \maketitle
    \newpage

    \tableofcontents

    \newpage

    \section{Teachers' notes}

    Each group should designate a reference user. In the local home directory of such user there must be an {\sf exam} folder exclusively containing the jobs to run (e.g., MapReduce jar, Spark scala file, Spark jar). Also, please send (either by email or by sharing a Git project) the following files.

    \begin{itemize}
        \item The source code of the jobs; if more versions have been developed, only send the most efficient one.
        \item A text file with the commands to run the job.
        \item The PDF file of the report; use Italian/English and Latex/Word at your discretion. Be concise and go straight to the point; do not waste time and space on writing a verbose report.
    \end{itemize}

    This guide is based on the ``MapReduce+Spark'' kind of project. However, we remind that a different kind of project may be agreed upon.
    \\

    The evaluation will be based on the following.
    \begin{itemize}
        \item Compliance of the jobs with the agreed upon specifications.
        \item Compliance of the report with this guide.
        \item Job correctness.
        \item Correct reasoning about optimizations.
    \end{itemize}

    Appreciated aspects.
    \begin{itemize}
        \item Code cleanliness and comments.
        \item Further considerations in terms of job scalability and extensibility.
    \end{itemize}



    \section{Introduction}
    \subsection{Dataset description}

    Please provide:
    \begin{itemize}
        \item A brief description of the dataset.
        \item The link to the website publishing the dataset (e.g., \url{https://www1.nyc.gov/site/tlc/about/tlc-trip-record-data.page}).
        \item Direct links to the downloaded files, especially if more than one files are available in the previous link (e.g., \url{https://s3.amazonaws.com/nyc-tlc/trip+data/yellow_tripdata_2017-01.csv}).
    \end{itemize}

    \subsubsection{File description}

    For each file, briefly indicated the available data and the fields used for the analyses; examples are welcome.


    \section{Data preparation}

    Please provide:
    \begin{itemize}
        \item The name of the reference user (i.e., the one in whose home directory is the {\sf exam} folder).
        \item The machine name (or IP address) of the reference user.
        \item The paths to each file on HDFS and/or its corresponding location in Hive (database and table); consider relying on the structured data lake organization.
        \item A subsection with details on the pre-processing of the data (only necessary if the data is dirty and/or it contains a significant amount of useless information).
    \end{itemize}


    \section{Jobs}

    One subsection for each job.

    \subsection{Job \#1: short description}

    Provide a brief, general description of the job. Then, one subsubsection for each implementation.

    \subsubsection{MapReduce/Spark(SQL) implementation}

    Please provide:
    \begin{itemize}
        \item The command to run the job from the reference user's home directory; explain possibly different parameter configurations.
        \item Direct link to the application's history on YARN (e.g., \url{http://isi-vclust0.csr.unibo.it:18088/history/application_15...}).
        \item Input files/tables.
        \item Output files/tables.
        \item Description of the implementation. A schematic and concise discussion is preferrable to a verbose narrative. Focus on how the data is manipulated in the job (e.g., what do keys and values represent across the different stages, what operations are carried out).
        \item Performance considerations with respect the (potentially) carried out optimizations, e.g., in terms of:
        \begin{itemize}
            \item allocated resources and tasks;
            \item enforced partitioning;
            \item data caching;
            \item combiner usage;
            \item broadcast variables usage;
            \item any other kind of optimization.
        \end{itemize}
        \item Short extract of the output and discussion (i.e., whether there is any relevant insight obtained).
    \end{itemize}

    \section{Miscellaneous}

    If necessary, feel free to add sections to explain any other relevant information.

\end{document}